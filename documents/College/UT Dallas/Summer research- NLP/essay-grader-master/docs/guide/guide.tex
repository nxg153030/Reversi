\documentclass[12pt]{article} \title{ Persuasive Essay Annotation
  Guide [DRAFT]}

\usepackage{geometry} \geometry{margin=1in}

\usepackage{tabularx} % nice table formatting
\usepackage{hyperref} \usepackage{backref}

\begin{document}
\maketitle
\label{toc}
\tableofcontents

\newpage

\section{Annotation Procedure}
\label{sec:annotation-procedure}
Annotation should be performed in a bottom-up/depth-first manner.
\begin{enumerate}
\item Read through the essay for comprehension and a holistic
  understanding of its persuasiveness.
\item For the first Claim, annotate all bottom-level Premises
  (Premises not supported by other Premises). For each Premise:
  \begin{enumerate}
  \item Annotate the Eloquence, Specificity, and PremiseType.
  \item Reread the parent entity and assign a Relevance score.
  \item Score the persuasiveness contribution of the Premise.
  \end{enumerate}
\item Annotate all Premises that are completely supported by other fully
  annotated Premises.
\item Repeat until all Premises of the first Claim are annotated.  
\item Annotate the first Claim.
  \begin{enumerate}
  \item Review the supporting Premises and their attributes.
  \item Score the Specificity, Evidence, and whether or not the author
    uses Logos, Pathos, or Ethos in the argument.
  \item Annotate the ClaimType and Eloquence.
  \item Score the Persuasiveness.
  \item Reread the MajorClaims and assign a Relevance score.
  \end{enumerate}
  % \begin{enumerate}
  % \item Annotate the Eloquence
  % \end{enumerate}
\item Repeat steps 2--5 for the remaining Claims.
\item Annotate the MajorClaims. For each MajorClaim:
  \begin{enumerate}
  \item Review the supporting Claims and their attributes.
  \item Assign an Evidence score.
  \item Determine whether or not the argument uses Logos, Pathos, or
    Ethos in the argument.
  \item Score the Specificity.
  \item Assign an Eloquence score
  \item Score the Persuasiveness.
  \end{enumerate}
  
\end{enumerate}

\section{Attributes Summary}
\label{sec:attributes-summary}
\begin{tabularx}{\textwidth}{|l|X|}
  \hline
  Name & Description \\ \hline % Entities & Desc. \\ \hline
  Persuasiveness:
       & The general persuasiveness of the argument. If the entity is a
         Premise, the Persuasiveness score indicates the
         contribution made to the parent argument. \\ \hline % & x & x & x \\ \hline
       & (1--6; Claim, MajorClaim, Premise) \\ \hline % & & & \\ \hline
       & Depends on: PremiseType, Ethos/Pathos/Logos, Relevance, Eloquence, Specificity\\ \hline
  Relevance:
       & The relevance of the statement to the parent entity. \\ \hline
       & (1--6; Claim, Premise)\\ \hline
       & Depends on: Eloquence and Specificity of Child and Parent entities, text \\ \hline
  PremiseType:
       & The type of Premise, e.g. statistics, definition, real
         example, etc. \\ \hline
       & (see section~\ref{sec:premise-type}; Premise)\\ \hline
       & Depends on: text \\ \hline
  Logos/Pathos/Ethos:
       & Whether or not the argument uses the respective persuasive strategy \\ \hline
       & (yes,no; MajorClaim, Claim) \\ \hline
       & Depends on: text, child entities\\ \hline
  Evidence:
       & The holistic summary of the supporting statements making
         up the argument body of that specific entity \\ \hline
       & (1--6; Claim, MajorClaim) \\ \hline
       & Depends on: child entities \\ \hline 
  ClaimType:
       & The category of what is being claimed \\
       & (Value, Fact, Policy; Claim) \\ \hline
       & Depends on: Text \\ \hline
  Specificity:
       & How detailed and specific the statement is.\\ \hline
       & (1--5; Claim, MajorClaim, Premise) \\ \hline
       & Depends on: Text \\ \hline
  Eloquence:
       & How well the idea is presented. Influenced by grammar,
         vocabulary, and sentence structure. \\ \hline 
       & (1--5; Premise) \\ \hline
       & Depends on: Text \\ \hline
\end{tabularx}

\section{Entity Summary}
\label{sec:entity-summary}
\begin{tabularx}{\textwidth}{r X}
  
  \textbf{MajorClaim} - & Persuasiveness, Logos/Pathos/Ethos, Evidence,
                          Eloquence, Specificity \\
  \textbf{Claim} - & Persuasiveness, ClaimType, Relevance, Logos/Pathos/Ethos, Evidence,
                     Eloquence, Specificity \\
  \textbf{Premise} - & Persuasiveness, PremiseType, Relevance,
                       Eloquence, Specificity \\
\end{tabularx}

\pagebreak

\section{Eloquence}
\label{sec:eloquence}

Eloquence Score descriptions
% TODO: Provide example statements

% TODO: Grammatical errors are not the only factor. Include vocabulary
% and readability in every score descriptor
\noindent
\begin{tabularx}{\textwidth}{|l|X|}
  \hline
  Score & Description \\ \hline
  5 & Demonstrates mastery of the English language. There are no
      grammatical errors that distract from the meaning of the
      sentence. Exhibits a well thought out, flowing sentence
      structure that is easy to read and conveys the idea
      exceptionally well.\\ \hline
  {} & Ex: Contrary to the past when people had to wait long hours to
       take a daily newspaper, nowadays, they can acquire latest news
       updated every second through their mobile phones or computers
       connected to the internet, everywhere and at anytime \\ \hline 
  4 & Demonstrates fluency in English. If there are any grammatical or
      syntactical errors, their affect on the meaning is
      negligible. Word choice suggests a broad vocabulary.\\ \hline 
  {} & Ex:  \\ \hline
  3 & Demonstrates competence in the English language. There might be one
      or two errors that are noticeable but forgivable, such as an
      incorrect verb tense or unnecessary pluralization. Demonstrates
      a typical vocabulary and a simple sentence structure.\\ \hline 
  {} & Ex:  \\ \hline
  2 & Demonstrates poor understanding of sentence composition and/or poor
      vocabulary. The choice of words or grammatical errors force the
      reader to reread the sentence before moving on. \\ \hline 
  {} & Ex:  \\ \hline
  1 & Demonstrates minimal eloquence. The sentence contains errors
      so severe that the sentence must be carefully analyzed to
      deduce its meaning.\\ 
  \hline
  {} & Ex:  Health education will help people to have right choices
       of treatment to recover more quickly \\ \hline
\end{tabularx}

\pagebreak

\section{Specificity}
\label{sec:specificity}
Specificity score descriptions and examples.
% TODO: fit table on one page

\subsection{Premise Specificity}
\noindent
\begin{tabularx}{\textwidth}{|l|X|}
  \hline
  Score & Description \\
  \hline
  5 & A score of 5 indicates an elaborate, very specific
      statement. The statement contains numerical data, or a
      historical example from the real world. There
      is both:
      \begin{itemize}
        \itemsep-3mm
      \item a sufficient qualifier indicating a the extent to which
        the statement holds true
      \item an explanation of why the statement is true
      \end{itemize}
      or:
      \begin{itemize}
      \item at least one real world example
      \end{itemize}
      or:
      \begin{itemize}
      \item a sufficient description of a hypothetical situation that
        would evoke a mental image of the situation in the minds of
        most readers
      \end{itemize}
  \\
  \hline {} & Ex: Take Thailand for example, in the Vietnam War, many
              American soldiers came to Thailand for a break and
              involved in sexual and drug activities, these huge
              demands caused many local businesses opened and
              expanded, even illegally involved
              in under-age prostitutes to maximize their profits. \\
  \hline 4 & A score of 4 indicates a more specific statement. It is
             characterized by either an explanation of why the
             statement is true, or a qualifier indicating when/to what
             extent the statement is true. Alternatively, it may list
             examples of items that do not qualify as historical events.\\
  \hline {} & Ex: Compared to the peers studying in the home country,
              it will be more likely for the one who is living
              overseas to be successful in adapting himself/herself
              into new environments
              and situations in life \\
  \hline {} & Ex: Before email and mobile phone, human beings
              communicated by meeting directly, sending letters or
              later, calling from home phones  \\
  \hline 3 & A score of 3 indicates a sufficiently specific
             statement. It simply states a relationship or a fact with
             little ambiguity.  \\ 
  \hline {} & Ex: employers are mostly looking for people who have
              international and language skills \\
  \hline {} & Ex: They can learn about their way of thinking, their
              customs
              and traditions, and way of life. \\
  \hline 2 & A broad statement. A statement with weak qualifiers such
             as ``maybe,'' ``might,'' and ``sometimes,'' will receive
             a score of 2 without other redeeming factors such as
             explicit examples, or elaborate reasoning. Additionally,
             there are few adjectives
             or adverbs. \\
  \hline
  {} & Ex: students might face many challenges in the host country\\
  \hline 1 & An extremely broad statement. There is no underlying
             explanation, qualifiers, or real-world examples. \\
  \hline
  {} & Ex: Learning about others' cultures is so important. \\
  \hline
\end{tabularx}

\subsection{Claim/MajorClaim Specificity}

% TODO: provide examples
% TODO: assign blame to low clarity/eloquence/specificity
% Low child relevance with high specificity assigns blame to child
% entities

% Low child relevance with low specificity assigns blame to parent entity

\noindent
\begin{tabularx}{\textwidth}{|l|X|}
  \hline
  Score & Description \\ \hline
  5 & Summarizes the argument well and has a qualifier that
      indicates the extent to which the claim holds true. Claims that
      summarize the argument well must reference most or all of the
      supporting entities. \\ \hline
  4 & A Claim that scores a 4 must summarize the argument very well by
      mentioning most or all of the supporting entities, but does not
      have a qualifier indicating the conditions under which the claim
      holds true. Alternatively, the Claim may moderately summarize
      the argument by referencing a minority of supporting entities and
      contain qualifier. \\ \hline 
  3 & A Claim with specificity 3 has a qualifier clause or references
      a minority of the supporting entities, but not both. \\
  \hline
  2 & A Claim with specificity 2 does not make an attempt to summarize
      the argument nor does it contain a qualifier clause. \\ \hline
  1 & Simply rephrases the Majorclaim or is outside scope of the Majorclaim (entities were annotated incorrectly: Majorclaim could be used to support Claim). \\ \hline
\end{tabularx}
\noindent
Note: A MajorClaim cannot receive a score of 1. \\
\noindent
Note: Some MajorClaims have an argument summary labeled as an
independent claim with no supporting attributes.
\pagebreak

\section{Premise Type}
\label{sec:premise-type}

\begin{tabularx}{\textwidth}{|l|X|}
  \hline
  Type & Description \\ \hline
  real\_example & A historical example of something that actually
                  happened, or a specific, non-generic, statement that is verifiably true
                  about the real world.\\ \hline
  {} & Ex: \\ \hline
  {} & Ex: \\ \hline
  invented\_instance & A hypothetical situation that did not actually occur.\\ \hline
  {} & Ex: Take Olympic games which is a form of competition for
       instance, it is hard to imagine how an athlete could win the
       game without the training of his or her coach, and the help of
       other professional staffs such as the people who take care of
       his diet, and those who are in charge of the medical care \\
  \hline 
  analogy &  \\ \hline
  testimony & A quote from, or reference to a higher authority. \\ \hline
  statistics & Raw numerical data or a quantitative comparison of
               values.\\ \hline 
  definition & An explicit definition of a term/concept.\\ \hline
  common\_knowledge & A conjecture or generalization that the author
                      assumes most people would accept as true. \\
  \hline 
  warrant & Performs one of the following functions:
            \begin{itemize} 
            \item Restates the parent claim
            \item expands/upon clarifies the parent claim
            \item explains how a sibling premise relates to the parent
              claim.
            \end{itemize}
  \\ \hline
  opinion & A controversial
            statement that would be unanimously accepted as an
            opinion. Could easily be the thesis of its own persuasive
            essay. \\ \hline
\end{tabularx}

\pagebreak

\section{Claim Type}
\label{sec:claimtype}
% TODO add examples
\noindent
\begin{tabularx}{\textwidth}{|l|X|}
  \hline
  Type & Description \\ \hline
  Fact & The claim states that something is true or false. \\ \hline
  {} & Ex: \\ \hline
  Value & The claim states that something is important, or not
          important, or has some other value attached to it. \\ \hline
  {} & Ex: \\ \hline
  Policy & The claim states that a certain law or rule should be
           implemented and enforced. \\ \hline
  {} & Ex: \\ \hline
\end{tabularx}

\pagebreak

\section{Relevance}
\label{sec:relevance}
Relevance score descriptions.\\
% TODO: Assign blame to unclear parent or child entity

\noindent
\begin{tabularx}{\textwidth}{|l|X|}
  \hline
  Score &  Description \\ \hline
  6 & Anyone can see how the support relates to the parent claim. The
      relationship between the two entities is either explicit or
      extremely easy to infer. The relationship is thoroughly
      explained in the text because the two entities contain the same
      words or exhibit coreference.\\
      \hline
  {} & Premises that are paired with a warrant
       sufficiently explaining the relationship to the parent claim (and the
       warrant itself) score a 6. If the warrant does a poor job of
       this, then the Premise and the Warrant receive the same, lower score.
       % If the entity in
       % question and its parent entity both have high clarity scores, so
       % it is easy to see how they relate.
  \\ \hline 
  5 & There is an implied relationship that is obvious, but it could
      be improved upon to remove all doubt. If the relationship is
      obvious, both relating entities must have high eloquence and specificity
      scores. \\ \hline  
  4 & The relationship is fairly clear. The relationship can be
      inferred from the context of the two statements. One entity must
      have a high eloquence and specificity scores and the other must have lower but
      sufficient eloquence and specificity scores for the relationship
      to be fairly clear.\\ \hline 
  3 & Somewhat related. It takes some thinking to imagine how the
      entities relate. The parent entity or the child entity have low
      clarity scores. Two statements about the same topic but unrelated
      ideas within the domain of said topic would get a score of 3. \\ \hline  
  2 & Mostly unrelated. It takes some major assumptions to relate the
      two entities.  An
      entity may also receive this score if both entities have low
      clarity scores.\\ \hline  
  1 & Totally unrelated. Very few people could see how the two
      entities relate to each other. The statement was annotated to
      show that it relates to the claim, but this was clearly in
      error.\\ \hline
\end{tabularx}

\pagebreak

\section{Persuasiveness}
\label{sec:persuasiveness}
Argument strength score descriptions.\\
% TODO: Explain relationship between persuasiveness and other attributes}
% TODO: Explain meaning of ``persuasiveness'' w.r.t. premises
\noindent
\begin{tabularx}{\textwidth}{|l|X|}
  \hline
  Score & Description \\ \hline
  6 & A very strong, clear argument.  It would persuade
      most readers and is devoid of errors that might detract from its
      strength or make it difficult to understand. \\ \hline
  Claim/MajorClaim  &  \\ \hline
  Premise & \\ \hline
  5 & A strong, pretty clear argument.  It would persuade
      most readers, but may contain some minor errors that detract from
      its strength or understandability. \\ \hline
  Claim/MajorClaim  &  \\ \hline
  Premise & \\ \hline
  4 & A decent, fairly clear argument.  It could persuade
      some readers, but contains errors that detract from its strength
      or understandability. \\ \hline
  Claim/MajorClaim & 
  \\ \hline
  Premise & \\ \hline
  3 & A poor, understandable argument.  It might persuade
      readers who are already inclined to agree with it, but contains
      severe errors that detract from its strength or
      understandability. \\ \hline
  Claim/MajorClaim  & 
  \\ \hline
  Premise & \\ \hline
  2 & It is unclear what the author is trying to
      argue or the argument is poor and just so riddled with errors as
      to be completely unpersuasive. \\ \hline
  Claim/MajorClaim  & 
  \\ \hline 
  Premise & \\ \hline
  1 & The author doesn't appear to make any argument 
      (e.g. he may just describe some incident without explaining why
      it is important).  It could not persuade any readers because
      there is nothing to be persuaded of.  It may or may not contain
      detectable errors, but errors are moot since there is not an
      argument for them to interfere with. \\ \hline
  Claim/MajorClaim;  &  \\ \hline
  Premise & \\ \hline
\end{tabularx}
Note: Persuasiveness is a property specific to arguments. Premises and
claims by themselves do not constitute arguments, so the
``persuasiveness'' attribute has a slightly different meaning than the
name implies. With respect to claims and majorclaims, the
``persuasiveness'' score refers to the argument corresponding to the
claim.



\pagebreak

\section{Evidence}
\label{sec:evidence}
% TODO: explain how eloquence affects evidence
6 - Indicates a very strong, very persuasive argument body. There are
many supporting entities that have high relevance scores.

There may be a few  attacking child entities. These entities
must be used for either
\begin{itemize}
\item Concession: acknowledging valid points on the opposing side
  
  A child entity successfully used for a concession will itself have a
  strong Persuasiveness score. An entity will have a high Evidence
  score if there is a small number of concessions with good
  persuasiveness. These concessions do not significantly detract from
  the original arguments strength. They enhance persuasiveness by
  revealing the author's knowledge of the subject and show that the
  author is aware that the Claim/MajorClaim may not be absolute due
  to the complexity and subjectivity of the real world.
\item Refuting Counterarguments: anticipating potential
  counterarguments and explaining why they are not valid
  
  An entity containing successfully refuted child entities that attack
  the Claim/MajorClaim can receive an Evidence score of 6. If
  successfully refuted, the opposing child entities will have a low
  Persuasiveness score. If the opposing child entities are unrefuted
  and carry persuasive power, they will detract
   
\end{itemize}
The entities must perform one of the above functions as opposed to
making the argument indecisive or contradictory. Only then can the
Evidence score a 6.
\begin{itemize}
\item \textit{MajorClaim} \\ This score implies a high quantity of
  supporting claims with high persuasiveness and relevance
  scores. There are no attacking claims, or there are few attacking
  claims with low persuasiveness, for the purpose of refuting potential
  counterarguments.
\item \textit{Claim} \\ This means a diverse selection of high-quality
  premise types (e.g. real example, statistics, etc.) and high
  relevance. There might be one "opinion" premise type or a premise
  with low relevance, but the argument body is still strong
  regardless.
\end{itemize}
5 - Indicates a strong, persuasive argument body. There are sufficient
supporting entities with respectable scores.
\begin{itemize}
\item \textit{MajorClaim} \\ Indicates supporting entities for this
  argument have high relevance and high persuasive scores. There may
  be a conflicting claim, but its score is so low that it does not
  hurt the persuasiveness of the MajorClaim in question.
\item \textit{Claim} \\ There are numerous high quality
  premises. PremiseTypes are mostly homogeneous and are not varied.
  There might be enough premises with low relevance to distract from
  the subject, thereby detracting from the persuasiveness.
\end{itemize}
4 - Indicates a decent, fairly persuasive argument body.
\begin{itemize}
\item \textit{MajorClaim} \\ There is a sufficient number of
  supporting claims to create a persuasive argument. The supporting
  claims have high relevance scores, and decent persuasiveness on
  their own. There might be a few strongly persuasive claims that take
  a stance against the MajorClaim.
\item \textit{Claim} \\ There is a small number of premises with
  strong premise types and high relevance. Or maybe there are many
  premises with too many low quality premises to be considered a very
  strong, persuasive argument.
\end{itemize}
3 - Indicates a poor, possibly persuasive argument body.
\begin{itemize}
\item \textit{MajorClaim} \\ The supporting claims were unpersuasive
  and/or received low relevance scores. A MajorClaim that would have
  otherwise received a 5 or 6 can get reduced to a 3 if the strength
  of the claims against it approach the strength of those supporting
  it.
\item \textit{Claim} \\ A claim with an Evidence score of 3 means that
  there is a mix of persuasive premise types with decent relevance,
  and premise types that are weak with low relevance.
\end{itemize}
2 - Indicates a totally unpersuasive argument body.
\begin{itemize}
\item \textit{MajorClaim} \\ All of the supporting claims had low
  persuasiveness scores, and/or low relevance. There are one or more
  strong, unrefuted Claims against it.
\item \textit{Claim} \\ Most, if not all of the supporting premises
  were opinions, and/or the majority of the premises had a low
  relevance score. There are a small number of supporting
  premises. Also, a claim will receive a score of 2 if there are too
  many opposing, unrefuted premises.
\end{itemize}

\noindent
A score of 1 indicates that there is no argument body for the given entity.
\begin{itemize}
\item \textit{MajorClaim} - No supporting claims.
\item \textit{Claim} - No supporting premises.
\end{itemize}

\pagebreak

\section{Example Walkthrough}
\label{sec:example-walkthrough}
% TODO: Step by step annotation of an example essay

% \section{References}
\end{document}
%%% Local Variables:
%%% mode: latex
%%% TeX-master: t
%%% End:
